\documentclass[letterpaper,titlepage,12pt,spanish]{article}
\usepackage[utf8]{inputenc}
\usepackage{times}
\usepackage{hyperref}
\usepackage[spanish]{babel}
\usepackage{graphics}
\usepackage{graphicx}
\usepackage[]{graphicx}
\usepackage{enumerate}
\usepackage{verbatim}
\usepackage{float}
\usepackage[spanish]{layout}
\usepackage{latexsym}
\usepackage[scriptsize,it]{caption}
\usepackage{fancyhdr}
\setlength{\headheight}{15pt}
\setlength{\headsep}{14pt} %OJO
\setlength{\footskip}{26pt} %OJO
\pagestyle{fancy}
\usepackage{colortbl}
\usepackage{multirow}
\hyphenation{EAFIT}
%\pagestyle{empty}
\sloppy
\hypersetup {bookmarksopen,bookmarksnumbered,colorlinks,linkcolor=blue,legalpaper,pdftitle=Libro de proyecto,pdfauthor=Carlos Urrego Moreno Juan David Pineda Cardenas Santiago Toro Acelas Héctor Espoz Alvarez}

\title{Centro de Computación Científica APOLO \\ Informe de Gestión 2013}
\vspace{1cm}
\author{Juan Guillermo Lalinde--Pulido\\jlalinde@eafit.edu.co \and Juan David Pineda--Cárdenas\\jpineda2@eafit.edu.co}
\date{
	\includegraphics[width=0.35\textwidth]{imgs/logo_apolo}\\[2cm]
	Universidad EAFIT\\
	Medellín, Colombia
}
\begin{document}
\fancyhead[]{}
\fancyhead[C]{Centro de Computación Científica Apolo}
\maketitle
%\begin{abstract}
%Este documento hace un breve resumen acerca del funcionamiento y mejoras realizados durante el año 2013 en el Centro de Computación Científica APOLO.
%\end{abstract}
\thispagestyle{empty}
\tableofcontents
\pagenumbering{arabic}
\newpage
Dentro de los problemas que se estudian en la Ingeniería de Sistemas y sus aplicaciones existe una gran cantidad de problemas que se han planteado para resolver, continuamente incrementa esta cantidad y  aun mas complejos, con mas interacciones y mas datos para analizar. 

Entre estos existen algunos con una complejidad alta, donde de la solución de los mismos esta condicionada a mucho tiempo de computo, haciendo de tales investigaciones, trabajos y simulaciones algo inviable por su prolongada ejecución. Con el crecimiento de la capacidad computacional y los algoritmos paralelos ya es posible abordar algunos de los problemas que antes hubieran tomado mucho tiempo o simplemente no hubiese sido posible resolver, pasando de semanas, meses y años de computo, a solo semanas o incluso días.  

De este conjunto de problemas, existen un subconjunto particular los cuales pueden ser modelados mediante la teoría de grafos, una  abstracción  que ofrece las ciencias de la computación para la solución de un conjunto determinado de problemas que cumplen con ciertas características, y aunque los algoritmos para grafos tienen una complejidad alta, existe la posibilidad de utilizar algoritmos paralelos.

En este proyecto se pretende introducir al lector en el amplio mundo de la computación de alto rendimiento, mediante el procedimiento de instalación y ejecución en un ambiente de pruebas que pretende simular el funcionamiento del centro de computación científica ``Apolo'', el supercomputador de la Universiad EAFIT. La configuración y uso de la ``infraestructura'' básica para que un programador, utilizando las librerías BOOST para C++, pueda desarrollar aplicaciones que aprovechen su capacidad computacional. 

\section{Justificación}

Dada la configuración de Sistema necesaria para la realización de este proyecto, los procedimientos y productos aquí planteados pueden ser de utilidad en diferentes campos y aplicaciones, entre ellas destacan las siguientes:  

\begin{itemize}
	
	\item Personas que requieran una introducción a la configuración de un sistema de pruebas para la computación paralela, en particular una simulación de la arquitectura y sistema utilizado por el supercomputador de la universidad EAFIT.

	\item En el mundo académico también puede ser importante para Investigación, Estudio y análisis de la Arquitectura de un clúster a través de un sistema virtualizado, con diferentes configuraciones de redes implementadas en el clúster, para analizar rendimiento, velocidad, eficiencia y diferentes características importantes. 


	\item Análisis y practicas de telemática y comunicación en el clúster. 

	\item Toda aquella persona que necesite un ambiente de pruebas de un clúster de computación de tipo HPC y HTC.

	\item Quien necesite probar el uso de algoritmos con MPI, Boost u otro ambiente de computación en paralelo.

	\item Algoritmos para el procesamiento de Grafos en paralelo (tales como PBGL).

	\section{Posibles Aplicaciones}

	Como posibilidades para la aplicación de este proyecto se mencionan las siguientes áreas: 

	\begin{itemize}
		\item En la Industria: Análisis de malla vial de la ciudad a través de un grafo, análisis de rutas, vías alternas en caso de cierre de vías, camino más corto entre dos puntos, optimización de recorrido para ambulancias, bomberos, policías o atención de emergencias, análisis de la estructura de distribución de mallas eléctricas, análisis de la estructura de distribución de la infraestructura hídricas.

		% \begin{itemize}
		% 	\item Análisis de Malla vial de la ciudad a través de un grafo:
		% 		\begin{itemize}
		% 		 	\item Análisis de Rutas.
		% 		 	\item Vías alternas (en caso de cierre de vías).
		% 		 	\item Camino más corto entre dos puntos. 
		% 		 	\item Optimización de recorrido para ambulancias, bomberos, policías o atención de emergencias. 
		% 		 \end{itemize} 

		% 	\item Análisis de la estructura de distribución de mallas eléctricas 
		% 	\item Análisis de la estructura de distribución de la infraestructura hídricas.

		% \end{itemize}

		\item Para Investigación: Simulaciones de mecánica de materiales, simulaciones para Ingeniería Civil, optimización de redes, patrones de distribución de la información.

		% \begin{itemize}
		% 	\item Simulaciones de Mecánica de materiales.
		% 	\item Simulaciones para Ingeniería Civil.		
		% \end{itemize}

		\item Para Educación: Aplicación en la enseñanza en materias de algoritmia y los algoritmos paralelizados, 		enseñanza de seguridad informática (Criptografía).


			% \begin{itemize}
			% 	\item Aplicación en la enseñanza en materias de algoritmia y los algoritmos paralelizados.
			% 	\item Enseñanza de Seguridad Informática(Criptografía).
			% \end{itemize}

	\end{itemize}

		Los usuarios potenciales de este proyecto son: comunidad Científica y Académica, interesados en el desarrollo de software paralelo, simulaciones de diferentes tipos, investigación en diferentes campos que requieran la ejecución de software en paralelo,  simulaciones o procesos de la industria, estudiantes de pregrado de instituciones académicas, estudiantes de posgrado de instituciones académicas. 

		% \begin{itemize}
		% 	\item 
		% 	\item 
		% 	\item 
		% 	\item 
		% 	\item 
		% 	\item 
		% 	\item 
		% \end{itemize}

\end{itemize}


\section{Objetivos}

	\begin{itemize}
		\item Generar documentación sobre el manejo de las librerías de Boost.
		\item Construir un manual para la instalación de las librerías de Boost.
		\item Correr códigos de ejemplo que hagan uso de las librerías Paralelas de Boost para el recorrido de grafos.	
	\end{itemize}

\section{Alcance}
	\begin{itemize}
		\item Introducción y descripción de las librerías de Boost.
		\item Introducción a las librerías de utilización de grafos a ser instaladas.
		\item Introducción al manejo de grafos con Boost (representación conceptual).
	\end{itemize}

\section{Productos}
	\begin{itemize}
		\item Manual de creación de un ambiente de pruebas para la instalación de las librerías de Boost.
		\item Manual de instalación de librerías Para Boost.
		\item Manual de ejecución para el uso de las librerías de Boost para el grafos en paralelo.	
	\end{itemize}


\end{document}

