
\section{Cómo Usar el Sistema de Colas}
    \subsection{Trabajos secuenciales}
    \subsubsection{Ejemplo de Archivo de Torque para trabajo secuencial}
    \begin{verbatim}
    #!/bin/bash
    #PBS -N NombreTrabajo
    #PBS -l nodes=1
    #PBS -l walltime=00:10:00
    #PBS -q standby
    #PBS -M correo@dominio
    #PBS -m abe

    cd $PBS_O_WORKDIR

    ./unejecutable
    \end{verbatim}

    \subsection{Trabajos en paralelo y distribuidos}
    \subsubsection{Ejemplo de Archivo de Torque para trabajo paralelo o distribuido}
    \begin{verbatim}
    #!/bin/bash
    #PBS -N NombreTrabajo
    #PBS -l nodes=1:ppn=8
    #PBS -l walltime=36:00:00
    #PBS -q longjobs
    #PBS -M correo@dominio
    #PBS -m abe

    cd $PBS_O_WORKDIR

    nprocs=`cat $PBS_NODEFILE | wc -l`

    mpirun -np $nprocs -machinefile $PBS_NODEFILE ejecutableMPI
    \end{verbatim}
\section{Cómo Verificar que mi simulación esté funcionando bien}
\section{Torque}
\section{Errores}
\section{Resultados}

