\section{Uso del Supercomputador}

\subsection{Ingreso al Supercomputador}
\subsection{Cómo usar la VPN y en que casos}

https://hiperion.eafit.edu.co/apolo

AnyConnect

\subsection{Tipo de software dentro de Apolo}
\subsection{Que pasa si el software no existe?}
\subsection{Cómo conectarse a Apolo}


\subsection{Procedimiento de Conexión hacia Apolo}


A continuación se especifícan los pasos a seguir para que un usuario externo a la Universidad EAFIT pueda acceder a los recursos computacionales ofrecidos por el centro de computación científica. 

\begin{enumerate}
\item Solicitar al administrador del sistema una cuenta en Apolo por medio del correo apolo@eafit.edu.co o contactar con el administrador en el teléfono +57 (4) 2619592\footnote{Debe tener en cuenta que este número telefónico se encuentra en la ciudad de Medellín, Colombia y pueden aplicar cargos en llamadas nacionales e internacionales.} 
\item Una vez verificado y definido el propósito y la validez para el uso del supercomputador, el usuario deberá enviar una IP Pública y fija a la misma cuenta de correo para permitir el acceso en el cortafuegos de la Universidad.
\item Una vez creada la cuenta, el usuario deberá verificar que tiene los permisos tanto desde el lado de la Universidad EAFIT como desde su institución para acceder al \textit{nodo maestro} de Apolo por medio del protocolo ssh. En un apartado posterior se explicará desde el punto de vista técnico como se realizará esta conexión desde los distintos sistema operativos. 
\end{enumerate}


\begin{itemize}
\item Windows
\item Linux
\item Mac OS X
\end{itemize}

\subsection{Uso del sistema de Colas}


\subsection{Ambiente de Pruebas}
\subsubsection{Instalación de las máquinas virtuales}
\subsubsection{Manejo e instalación de software en el ambiente de pruebas}

\subsection{Referencias}


\subsubsection{Como referenciar a Apolo en Publicaciones}

\todo[inline,caption={TODO}]{
¿Cómo citar a apolo? 
aramir21@eafit.edu.co
Pagina de Apolo. 
Documentación de Apolo. 
}


\subsubsection{Otros recursos}
	\begin{itemize}
	\item Apolo Hub
	\end{itemize}



\subsubsection{PuTTy}

network-manager-openconnect-gnome

pasarela(gateway) hiperion.eafit.edu.co/apolo

Cisco AnnyConnect

cmd  (Comand Line Promp )


\section{Conectividad}

Para interactuar con apolo es necesario hacer una conexion con el, configuración  a travez de los siguientes metodos: 


Putty  (linea de comandos)
FileZilla   (subir y bajar archivos)


\subsection{ssh}

\subsubsection{PuTTy}
PuTTY is an SSH and telnet client, developed originally by Simon Tatham for the Windows platform. PuTTY is open source software that is available with source code and is developed and supported by a group of volunteers.

\subsubsection{Pruebas de conectividad}

\section{Ejecución}


se conecta al master con un archivo de torque se le encargan las maquinas en las cuales seran ejecutadas las tareas
se envia un correo cuando comienza la simulación, cuando se completa, o cuando se aborta la simulacion

y luego se entra de nuevo a apolo para 



Servidor: Apolo.eafit.edu.co
usuario (usuario de apolo)
password(password de apolo)

Puerto 22



Carpetas de Apolo, donde poner programas, de donde sacar información,  

Instalación de Programas
Lista de programas que se encuentran en apolo



HostName: apolo.eafit.edu.co
Puerto 22


Filezilla

Servidor: apolo.eafit.edu.co
Usuario: (usuario dentro de apolo)  
Contraseña: (contraseña de usuario apolo)
Puerto: 22
Conexion Rapida
Remember