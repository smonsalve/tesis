\section{Justificación}

Dada la configuración de Sistema necesaria para la realización de este proyecto, los procedimientos y productos aquí planteados pueden ser de utilidad en diferentes campos y aplicaciones, entre ellas destacan las siguientes:  

\begin{itemize}
	
	\item Personas que requieran una introducción a la configuración de un sistema de pruebas para la computación paralela, en particular una simulación de la arquitectura y sistema utilizado por el supercomputador de la universidad EAFIT.

	\item En el mundo académico también puede ser importante para Investigación, Estudio y análisis de la Arquitectura de un clúster a través de un sistema virtualizado, con diferentes configuraciones de redes implementadas en el clúster, para analizar rendimiento, velocidad, eficiencia y diferentes características importantes. 


	\item Análisis y Practicas de Telemática y comunicación en el clúster. 

	\item Toda aquella persona que necesite un ambiente de pruebas de un clúster de computación  de tipo HPC y HTC.

	\item Quien necesite probar el uso de algoritmos con MPI, Boost u otro Ambiente de computación en paralelo.

	\item Algoritmos para el procesamiento de Grafos en paralelo (tales como PBGL).

	\newpage

	\section{Posibles Aplicaciones}

	Como posibilidades para la aplicación y trabajo futuro de este proyecto se destacan las siguientes: 

	\begin{itemize}
		\item En la Industria:

		\begin{itemize}
			\item Análisis de un grafo con la de Malla vial de la ciudad.
				\begin{itemize}
				 	\item Análisis de Rutas.
				 	\item Vías alternas (en caso de cierre de vías).
				 	\item Camino mas corto entre dos puntos. 
				 	\item Optimización de recorrido para ambulancias, bomberos, policías o atención de emergencias. 
				 \end{itemize} 

			\item Análisis de la estructura de distribución de mallas eléctricas 
			\item Análisis de la estructura de distribución de la infraestructura hídricas.

		\end{itemize}

		\item Para Investigación: 

		\begin{itemize}
			\item Simulaciones de Mecánica de materiales.
			\item Simulaciones para Ingeniería Civil.		
		\end{itemize}

		\item Para Educación: 
			\begin{itemize}
				\item Aplicación en la enseñanza en materias de algoritmia y los algoritmos paralelizados.
				\item Enseñanza de Seguridad Informática(Criptografía).
			\end{itemize}

	\end{itemize}

	Por lo tanto los usuarios potenciales de este proyecto son: 

		\begin{itemize}
			\item Comunidad Científica y Académica.
			\item Desarrollo de Software paralelo. 
			\item Simulaciones.
			\item Investigación en diferentes campos que requieran la ejecución de Software en Paralelo. 
			\item Simulaciones o procesos de la Industria.
			\item Estudiantes de Pregrado de instituciones académicas.
			\item Estudiantes de Posgrado de instituciones académicas.
		\end{itemize}

\end{itemize}

\newpage

\section{Objetivos}

	\begin{itemize}
		\item Hacer un estudio de la computación en paralelo, los clústers de computo científico. 
		\item Aprender sobre el manejo y uso  de las librerías de Boost.
		\item Correr códigos de ejemplo que hagan uso de las librerías Paralelas de Boost para el análisis de grafos.
	\end{itemize}

\section{Alcance}
	\begin{itemize}
		\item Introducción y Descripción de las librerías de Boost.
		\item Introducción a las librerías de utilización de grafos a ser instaladas.
		\item Introducción al manejo de Grafos con Boost (representación conceptual).
	\end{itemize}

\section{Productos}
	\begin{itemize}
		\item Manual de creación de un ambiente de pruebas para la instalación de las librerías de Boost.
		\item Manual de Instalación de librerías Para Boost.
		\item Manual de Ejecución y Código de Ejemplo para el uso de las librerías de Boost para el grafos en paralelo.	
	\end{itemize}

