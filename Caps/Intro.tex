\section{Justificación}



Dada la configuración de Sistema necesaria para la realización de este proyecto, los procedimientos y productos aquí planteados pueden ser de utilidad en diferentes campos y aplicaciones, entre ellas destacan las siguientes:  

\begin{itemize}
	
\item Personas que requieran una introducción a la configuración de un sistema de pruebas para la computación paralela, en particular una simulación de la arquitectura y sistema utilizado por el supercomputador de la universidad EAFIT.


\item En el mundo académico también puede ser importante para Investigación, Estudio y análisis de la Arquitectura de un clúster a través de un sistema virtualizado, con diferentes configuraciones de redes implementadas en el clúster, para analizar rendimiento, velocidad, eficiencia y diferentes características importantes. 


\item Análisis y Practicas de Telemática y comunicación en el clúster. 

\item Toda aquella persona que necesite un ambiente de pruebas de un clúster de computación  de tipo HPC y HTC.

\item Quien necesite probar el uso de algoritmos con MPI, Boost u otro Ambiente de computación en paralelo.

\item Algoritmos para el procesamiento de Grafos en paralelo (tales como PBGL).

Por lo tanto los usuarios potenciales de este proyecto son: 

\begin{itemize}
	\item Comunidad Científica y Académica.
	\item Desarrollo de Software paralelo. 
	\item Simulaciones.
	\item Investigación en diferentes campos que requieran la ejecución de Software en Paralelo. 
	\item Simulaciones o procesos de la Industria.
	\item Estudiantes de Pregrado.
	\item Estudiantes de Posgrado.
\end{itemize}

\section{Posibles Aplicaciones}

Como posibilidades para la aplicación y trabajo futuro de este proyecto se destacan las siguientes: 

\begin{itemize}
	\item En la Industria:

	\begin{itemize}
		\item Análisis de un grafo con la de Malla vial de la ciudad.
			\begin{itemize}
			 	\item Análisis de Rutas
			 	\item Vías alternas (en caso de cierre de vías)
			 	\item Camino mas corto 
			 \end{itemize} 

		\item Análisis de la estructura de distribución de mallas eléctricas o hídricas.

	\end{itemize}

	\item Para Investigación: 

	\begin{itemize}
		\item Simulaciones de Mecánica de materiales.
		\item Simulaciones para Ingeniería Civil.		
	\end{itemize}

	\item Para Educación: 
		\begin{itemize}
			\item Aplicación en la enseñanza en materias de algoritmos para la Paralelización de estos. 
			\item Enseñanza de Seguridad Informática
		\end{itemize}

\end{itemize}
\end{itemize}

\section{Objetivos}

	\begin{itemize}
		\item Construir un manual para la instalación de las librerías de Boost.
		\item Generar Documentación sobre el manejo de las librerías de Boost.
		\item Correr códigos de ejemplo para la ejecución de las librerías de Boost.
	\end{itemize}

\section{Alcance}
	\begin{itemize}
		\item Descripción de las librerías de Boost.
		\item Introducción a las librerías de utilización de grafos instaladas en Apolo.
		\item Introducción al manejo de Grafos con Boost (representación conceptual).
	\end{itemize}

\section{Productos}
	\begin{itemize}
		\item Manual de Instalación de librerías Para Boost.
		\item Ejemplo en código documentado y manual de como hacer un programa utilizando Boost para grafos.	
		\item 
	\end{itemize}

