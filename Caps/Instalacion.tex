Este procedimiento de instalación esta basado en la instalación de Repast HPC del cual podran ver mas en el

\todo[inline,caption={TODO}]{referencia de repast}

Para el proceso de instalación del software necesario, estando en el nodo maestro con el usario root procedemos a realizar las siguientes instrucciones: 

\todo[inline,caption={TODO}]{Crear archivo para instalación}

Debemos descargar GrafosApoloBoostMpi.tar.gz de la dirección \url{goo.gl/} y procedemos a descomprimirlo en la carpeta compartida del cluster

\begin{verbatim}
$ cd /share/Apps/installers/
\end{verbatim}

\begin{verbatim}
$ tar xvf GrafosApoloBoostMpi.tar.gz
\end{verbatim}

En este archivo se encuentran los siguientes archivos: 

\begin{itemize}
	\item Boost 1.54
	\item mpich 
	\item Archivo de instalación
\end{itemize}

\newpage

\section{Instalación MPI}

Para instalar mpich nos movemos a la carpeta de instalación:

\begin{verbatim}
$ cd /INSTALLATION
\end{verbatim}

y procedemos a ejecutar el script de instalación con mpich como argumento de la siguiente manera:

\begin{verbatim}
$ ./install.sh mpich
\end{verbatim}

el cual ejecutara la siguiente secuencia de instrucciónes: 

\begin{itemize}
	\item Define una variable como el directorio base donde se encuentran los archivos.
	\item Verfica que no se haya corrido anteriormente el scprit o instalado mpich.
	\item Verfica que se encuentre el archivo *.tar.gz para instalar mpich.
	\item Descomprime el archivo.
	\item Se cambia a la carpeta descomprimida.
	\item Configura con los argumentos de instalación.
	\item Hace make y make install.
	\item Exporta el path en el que quedo instalado.
\end{itemize}

\newpage

\lstinputlisting[label=Installing Mpich,caption=Script de Instalación de Mpich]{code/mpich.sh}

\newpage

\section{Instalación de Boost}

Para instalar boost nos movemos a la carpeta de instalación:

\begin{verbatim}
$ cd /INSTALLATION
\end{verbatim}

y procedemos a ejecutar el script de instalación con boost como argumento de la siguiente manera:

\begin{verbatim}
$ ./install.sh mpich
\end{verbatim}

el cual ejecutara la siguiente secuencia de instrucciónes: 

\begin{itemize}
	\item Define una variable como el directorio base donde se encuentran los archivos.
	\item Define una variable con la dirección que tiene mpich en el sistema.
	\item Verfica que no se haya instalado o corrido anteriormente boost.
	\item Verfica que se encuentre el archivo *.tar.gz para instalar boost.
	\item Descomprime el archivo.
	\item Se cambia a la carpeta descomprimida.
	\item Copia los archivos descomprimidos a la carpeta de boost.
	\item Configura con los argumentos de instalación y las banderas apropiadas (librerias a compilar).
	\item corre el script de compilación propio de boost 
\end{itemize}
	
\todo[inline,caption={TODO}]{Ambiente de Nota: la libreria que hace uso de Grafos en paralelo necesita de otras como : 
\begin{itemize}
	\item serialization 
	\item mpi 
	\item graph 
	\item graph parallel 
	\item regex 
\end{itemize}
}




\newpage
\lstinputlisting[label=Installing Boost,caption=Script de Instalación de Boost]{code/boost.sh}


\todo[inline,caption={TODO}]{	
}	


\begin{verbatim}
$ wget htop.rpm
\end{verbatim}


\begin{verbatim}
$ rocks run host rpm -ivh /export/apps/installers/*.rpm
\end{verbatim}
