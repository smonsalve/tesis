
Ejemplo Programa Corriendo en Apolo

Programa Normal. 

Ejecturar


Ejecución

Contando ya con el ambiente necesario para la ejecucion del codigo a paralelizar se procede a realizar los siguientes pasos: 


desde la maquina virtual del master abrimos una consola. 

nos loggeamos como el usuario sin privilegios a través de ssh

smonsalve@192.168.56.101

% mpirun -n 4 -machinefile nodes.txt breadth_first_search < weighted_graph.gr

%weighted_graph-bfs.dot


nodes.txt


compute-0-0
compute-0-0
compute-0-1
compute-0-1




Codigo Fuente

Datos de Entrada

Salida ( a donde? )

Archivo de Errores

Torque  ( Nodos  )

PBS




Instalacion de Boost




rocks run host rmpb -ivh /export/apps/installers/*.rpm


Agregar usuario sin privilegios. 

adduser smonsalve
passwd smonsalve
rocks sync users


insert-ethers --remove compute-0-0
insert-ethers --remove compute-0-1
insert-ethers 

rocks run host hostname
rocks run host poweroff



Debido a las relaciones de confianza entre el master y los nodo


ssh compute-0-0
ssh compute-0-1


Para monitorear la actividad del master y los nodos podemos utiliar el programa htop, el cual se encarga de mostrarnos los procesos, las cpus y los cores disponibles dentro del nodo. 
 