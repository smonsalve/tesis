\begin{itemize}

	\item Con este proyecto se realizó una serie de procedimientos para simular un Ambiente de Pruebas que replica la arquitectura del centro de computación científica Apolo. Para trabajar en  este ambiente de pruebas fue necesario aplicar los conocimientos adquiridos en materias como Sistemas Operativos, Arquitectura de Computadores, Lenguajes de Programación, Telemática, Estructuras de datos y Algoritmos, entre otras.  

	\item La Computación Paralela es un campo que en nuestro contexto tiene un gran mercado para su aplicación y este trabajo permite una introducción al tema, el cual se espera sea aprovechado por otras personas para su aplicación. 

	\item Las librerías de Boost son  reconocidas por su alta calidad, las cuales brindan elementos para el Usuario final que facilitan y potencian el trabajo realizado, permitiendo un análisis complejo de gran cantidad de datos, mediante diferentes herramientas, agilizando la implementación de tales proyectos con una alta calidad. 

	\item Las librerías de Boost mediante su implementación basada en la programación genérica permite hacer uso  de sus librerías de una manera fácil y transparente para el programador, permitiendo de esta manera una codificación clara y elegante, lo cual repercute en el tiempo de desarrollo y su legibilidad. 

	\item MPI proporciona un encapsulamiento para las librerías de Boost que permite el funcionamiento del sistema, de tal manera que es transparente para el usuario la comunicación entre toda la arquitectura. 

\end{itemize}