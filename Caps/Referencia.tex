
	\section{Computación Paralela}

	La computación paralela es el uso de multiples recursos computacionales para resolver un problema o una necesidad de computo en particular. La computación paralela surje como respuesta ante la necesidad de incrementar los recursos, sea en procesador, memoria y asi mejorar el tiempo de respuesta para problemas con alta complejidad computacional o alto volumen de analisis de datos. 

	El paradigma computacional tradicional ha sido de computación serial, donde una tarea es dividida en una serie finita de instrucciones que son ejecutada de forma secuencial, donde una sola instruccion es ejecutada en un momento dado. 

	La computación paralela rompe el paradigma anterior, buscando que en un momento dado se puedan ejecutar varias instrucciones, utilizando multiples procesadores y una entidad que orqueste los mismos.  

	\cite{PCI}
	\cite{SC}

	\section{Apolo}

	\section{Grafos}

	\section{C++}

	\section{MPI}

	\section{Boost}

		\begin{quotation}
		The Parallel Boost Graph Library is an extension to the Boost Graph Library (BGL) for parallel and distributed computing. It offers distributed graphs and graph algorithms to exploit coarse-grained parallelism along with parallel algorithms that exploit fine-grained parallelism, while retaining the same interfaces as the (sequential) BGL. Code written using the sequential BGL should be easy to parallelize with the parallel BGL. Visitors new to the Parallel BGL should read our architectural overview.\cite{wwwBoost} 
		\end{quotation} 


	\section{Parallel Boost Graph Library}

