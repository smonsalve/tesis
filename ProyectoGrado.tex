\documentclass[twoside,letterpaper,12pt]{report}
\usepackage[spanish,backgroundcolor=green,textsize=small]{todonotes}
\usepackage[margin=2cm]{geometry}
\usepackage[utf8x]{inputenc}
\usepackage[spanish]{babel}
\usepackage{plain}
\usepackage{amsmath}
\usepackage{graphicx}
\usepackage{wrapfig}
\usepackage{parskip}
\usepackage{nomencl}
\setlength{\parindent}{15pt}
\usepackage{hyperref}
\hypersetup{
  colorlinks=true,
  linkcolor=black,          % color of internal links (change box color with linkbordercolor)
  citecolor=black,        % color of links to bibliography
  filecolor=black,      % color of file links
  urlcolor=black           % color of external links }
}

\title{
	Proyecto de Grado\\[0.5cm]
	Instalación, Implementación y Uso de las librerías Boost y su aplicación en recorrido de grafos para la supercomputadora Apolo
}

\vspace{1cm}

\author{
	Autor:\\[0.3cm]
	Sergio Andrés Monsalve Castañeda\\
	Código: 200410061010\\
	smonsal3@eafit.edu.co\\[0.7cm]
	Asesor: \\[0.3cm]
	Juan Guillermo Lalinde Pulido\\
	jlalinde@eafit.edu.co\\
	2619500 ext 9588%\\[1cm]
}

\date{
	\today \\[0.5cm]
	\includegraphics[width=0.25\textwidth]{aux/logo_eafit} 
}

\begin{document}

% Primera Portada _____________________________________________________________________

\maketitle

% Segunda Portada _____________________________________________________________________

\thispagestyle{empty} % Esta pagina no se numera
\begin{center}
\textbf{{\Large Instalación, Implementación y Uso de las librerías Boost y su aplicación en recorrido de grafos para la supercomputadora Apolo}}\\[3cm]
{\Large Sergio Andrés Monsalve Castañeda} \\ {\large \textit{smonsal3@eafit.edu.co}}\\[2.5cm]
%{\large\emph{\textbf{Trabajo de grado presentado como requisito }}}\\
%\emph{\textbf{parcial  para optar al t\'itulo de Ingeniero Físico}}}\\[3cm]

{\large \textbf{Asesor:} \\ Doctor Juan Guillermo Lalinde Pulido}\\[3cm]

Ingeniería de Sistemas \\ Departamento de Informática y Sistemas  \\ Escuela de Ingeniería \\ Universidad EAFIT \\ Medellín, Colombia.\\
2013

\end{center}
\pagebreak

%------------------------- Aceptación --------------------------------%

\begin{flushright}
	Nota de aceptación\\
	\vspace{1cm}
	\rule{\textwidth}{1pt}\\
	\vspace{1cm}
	\rule{\textwidth}{1pt}\\
	\vspace{1cm}
	\rule{\textwidth}{1pt}\\
	\vspace{1.3cm}
	Presidente del jurado\\
	\vspace{1cm}
	\rule{\textwidth}{1pt}\\
	\vspace{1.3cm}
	Jurado\\
	\vspace{1cm}
	\rule{\textwidth}{1pt}\\
	\vspace{1.3cm}
	Jurado\\
	\vspace{1cm}
	\rule{\textwidth}{1pt}\\
	\vspace{2.5cm}
	\rule{\textwidth}{1pt}\\
	\vspace{.8cm}
	Ciudad y Fecha
\end{flushright}
\thispagestyle{empty}
\pagebreak


%---------------------- Tabla de contenido ---------------------------%

	\pdfbookmark[1]{\contentsname}{}
	\tableofcontents
	\cleardoublepage \phantomsection
	%\newpage
	\listoftables
	\addcontentsline{toc}{section}{\protect\numberline{}\listtablename}
	%\pdfbookmark[1]{Lista de tablas}{}
	\cleardoublepage \phantomsection
	%\newpage
	%\listfigurename
	\listoffigures
	\addcontentsline{toc}{section}{\protect\numberline{}\listfigurename}
	%\pdfbookmark[1]{Lista de figuras}{}
	\cleardoublepage \phantomsection
	%\newpage
	% \printnomenclature[2cm] %% Imprime la nomenclatura
	\addcontentsline{toc}{section}{\protect\numberline{}\nomname}
	%\pdfbookmark[1]{Lista de abreviaturas y símbolos}{}



\begin{abstract}

	Existe una gran cantidad de problemas que debido a su complejidad no es posible resolver, sea porque requieren mucho tiempo o la complejidad de estos esta tal que los hace inviables, adicionalmente cada vez surgen nuevos problemas, aun mas complejos. 

	Con el crecimiento de la capacidad computacional ya es posible abordar algunos de ellos que antes hubieran tomado mucho tiempo o simplemente no hubiese sido posible resolver. 

	Dentro de este amplia gama de problemas, existen un subcojunto particular que pueden ser modelados mediante la teoría de grafos, una gran herramienta que ofrece las ciencias de la computación.

	Aunque los algoritmos para grafos tienen una complejidad alta en general, existen algoritmos paralelos.

	En este proyecto se pretende instalar en Apolo, el supercomputador de EAFIT, la infraestructura básica para que un programador, utilizando las librerías BOOST para C++, pueda desarrollar rápidamente aplicaciones que aprovechen su capacidad computacional. 

	De esta manera se pretende beneficiar a la Comunidad Científica, Comunidad académica, Estudiantes de Ingeniería de Sistemas, Estudiantes de Ingenieria Matematica y en General a quien necesite realizar computo en un ambiente de computación de alto rendimiento, clusters computacionales, grafos en paralelo utilizando Boost, o requiera de una introducción a este tema.


\end{abstract}

\newpage

\renewcommand{\abstractname}{Agradecimientos}
\begin{abstract}

	En primer lugar quiero agradecer a Juan Guillermo Lalinde quien desde el inicio de mi carrera me inspiró y apoyó en grandes retos acádemicos. A Juan David Pineda por toda su paciencia para explicarme el funcionamiento de Apolo. Por ultimo agradezco a John Jairo Silva, Alejandro Gómez, Mateo Gómez, Jaime Pérez y John Mario Gutiérrez por sus multiples colaboraciones. 	

\end{abstract}

\newpage

\chapter{Introducción}
Dentro de los problemas que se estudian en la Ingeniería de Sistemas y sus aplicaciones existe una gran cantidad de problemas que se han planteado para resolver, continuamente incrementa esta cantidad y  aun mas complejos, con mas interacciones y mas datos para analizar. 

Entre estos existen algunos con una complejidad alta, donde de la solución de los mismos esta condicionada a mucho tiempo de computo, haciendo de tales investigaciones, trabajos y simulaciones algo inviable por su prolongada ejecución. Con el crecimiento de la capacidad computacional y los algoritmos paralelos ya es posible abordar algunos de los problemas que antes hubieran tomado mucho tiempo o simplemente no hubiese sido posible resolver, pasando de semanas, meses y años de computo, a solo semanas o incluso días.  

De este conjunto de problemas, existen un subconjunto particular los cuales pueden ser modelados mediante la teoría de grafos, una  abstracción  que ofrece las ciencias de la computación para la solución de un conjunto determinado de problemas que cumplen con ciertas características, y aunque los algoritmos para grafos tienen una complejidad alta, existe la posibilidad de utilizar algoritmos paralelos.

En este proyecto se pretende introducir al lector en el amplio mundo de la computación de alto rendimiento, mediante el procedimiento de instalación y ejecución en un ambiente de pruebas que pretende simular el funcionamiento del centro de computación científica ``Apolo'', el supercomputador de la Universiad EAFIT. La configuración y uso de la ``infraestructura'' básica para que un programador, utilizando las librerías BOOST para C++, pueda desarrollar aplicaciones que aprovechen su capacidad computacional. 

\section{Justificación}

Dada la configuración de Sistema necesaria para la realización de este proyecto, los procedimientos y productos aquí planteados pueden ser de utilidad en diferentes campos y aplicaciones, entre ellas destacan las siguientes:  

\begin{itemize}
	
	\item Personas que requieran una introducción a la configuración de un sistema de pruebas para la computación paralela, en particular una simulación de la arquitectura y sistema utilizado por el supercomputador de la universidad EAFIT.

	\item En el mundo académico también puede ser importante para Investigación, Estudio y análisis de la Arquitectura de un clúster a través de un sistema virtualizado, con diferentes configuraciones de redes implementadas en el clúster, para analizar rendimiento, velocidad, eficiencia y diferentes características importantes. 


	\item Análisis y practicas de telemática y comunicación en el clúster. 

	\item Toda aquella persona que necesite un ambiente de pruebas de un clúster de computación de tipo HPC y HTC.

	\item Quien necesite probar el uso de algoritmos con MPI, Boost u otro ambiente de computación en paralelo.

	\item Algoritmos para el procesamiento de Grafos en paralelo (tales como PBGL).

	\section{Posibles Aplicaciones}

	Como posibilidades para la aplicación de este proyecto se mencionan las siguientes áreas: 

	\begin{itemize}
		\item En la Industria: Análisis de malla vial de la ciudad a través de un grafo, análisis de rutas, vías alternas en caso de cierre de vías, camino más corto entre dos puntos, optimización de recorrido para ambulancias, bomberos, policías o atención de emergencias, análisis de la estructura de distribución de mallas eléctricas, análisis de la estructura de distribución de la infraestructura hídricas.

		% \begin{itemize}
		% 	\item Análisis de Malla vial de la ciudad a través de un grafo:
		% 		\begin{itemize}
		% 		 	\item Análisis de Rutas.
		% 		 	\item Vías alternas (en caso de cierre de vías).
		% 		 	\item Camino más corto entre dos puntos. 
		% 		 	\item Optimización de recorrido para ambulancias, bomberos, policías o atención de emergencias. 
		% 		 \end{itemize} 

		% 	\item Análisis de la estructura de distribución de mallas eléctricas 
		% 	\item Análisis de la estructura de distribución de la infraestructura hídricas.

		% \end{itemize}

		\item Para Investigación: Simulaciones de mecánica de materiales, simulaciones para Ingeniería Civil, optimización de redes, patrones de distribución de la información.

		% \begin{itemize}
		% 	\item Simulaciones de Mecánica de materiales.
		% 	\item Simulaciones para Ingeniería Civil.		
		% \end{itemize}

		\item Para Educación: Aplicación en la enseñanza en materias de algoritmia y los algoritmos paralelizados, 		enseñanza de seguridad informática (Criptografía).


			% \begin{itemize}
			% 	\item Aplicación en la enseñanza en materias de algoritmia y los algoritmos paralelizados.
			% 	\item Enseñanza de Seguridad Informática(Criptografía).
			% \end{itemize}

	\end{itemize}

		Los usuarios potenciales de este proyecto son: comunidad Científica y Académica, interesados en el desarrollo de software paralelo, simulaciones de diferentes tipos, investigación en diferentes campos que requieran la ejecución de software en paralelo,  simulaciones o procesos de la industria, estudiantes de pregrado de instituciones académicas, estudiantes de posgrado de instituciones académicas. 

		% \begin{itemize}
		% 	\item 
		% 	\item 
		% 	\item 
		% 	\item 
		% 	\item 
		% 	\item 
		% 	\item 
		% \end{itemize}

\end{itemize}


\section{Objetivos}

	\begin{itemize}
		\item Generar documentación sobre el manejo de las librerías de Boost.
		\item Construir un manual para la instalación de las librerías de Boost.
		\item Correr códigos de ejemplo que hagan uso de las librerías Paralelas de Boost para el recorrido de grafos.	
	\end{itemize}

\section{Alcance}
	\begin{itemize}
		\item Introducción y descripción de las librerías de Boost.
		\item Introducción a las librerías de utilización de grafos a ser instaladas.
		\item Introducción al manejo de grafos con Boost (representación conceptual).
	\end{itemize}

\section{Productos}
	\begin{itemize}
		\item Manual de creación de un ambiente de pruebas para la instalación de las librerías de Boost.
		\item Manual de instalación de librerías Para Boost.
		\item Manual de ejecución para el uso de las librerías de Boost para el grafos en paralelo.	
	\end{itemize}



\chapter{Marco de Referencia}\label{ChapRef}

	\section{Computación Paralela}

	La computación paralela es el uso de múltiples recursos computacionales para resolver un problema o una necesidad de computo en particular. La computación paralela surge como respuesta ante la necesidad de incrementar los recursos, sea en procesador, memoria y así mejorar el tiempo de respuesta para problemas con alta complejidad computacional o alto volumen de análisis de datos. 

	El paradigma computacional tradicional ha sido de computación serial, donde una tarea es dividida en una serie finita de instrucciones que son ejecutada de forma secuencial, donde una sola instrucción es ejecutada en un momento dado. 

	La computación paralela rompe el paradigma anterior, buscando que en un momento dado se puedan ejecutar varias instrucciones, utilizando múltiples procesadores y una entidad que orqueste los mismos.  

	Para mayor información sobre Computación Paralela \cite{PCI}

	\cite{SC}

	\section{Apolo}

	\todo[inline,caption={TODO}]{Arquitectura de apolo. Redes.}

	\section{Grafos}

	Los grafos son abstracciones matemáticas que son útiles a la hora de resolver muchos tipos de problemas en las ciencias de la computación. 
	Un grafo es un conjunto ordenado 

	Nodos

	Vertices

	donde los vértices son relaciones 


	La aplicación de la teoría de grafos estudio de relaciones entre nodos. 



	\begin{figure}[H]
		\centering
		\includegraphics[width=0.5\textwidth]{aux/grafo}
		\caption[Estructura de un Grafo]{
		(tomada de \cite{BoostGrafos)}
		%\label{F-dimensions-emotion}
	\end{figure}


	\begin{figure}[H]
		\centering
		\includegraphics[width=0.5\textwidth]{aux/distributed_graph}
		\caption[Grafo Distribuido]{
		(tomada de \cite{BoostGrafos)}
		%\label{F-dimensions-emotion}
	\end{figure}
	
	

	\begin{figure}[H]
		\centering
		\includegraphics[width=0.5\textwidth]{aux/dist-adjlist}
		\caption[Grafo en lista de adyacencia]{
		(tomada de \cite{BoostGrafos)}
		%\label{F-dimensions-emotion}
	\end{figure}


	\begin{figure}[H]
		\centering
		\includegraphics[width=0.5\textwidth]{aux/arquitectura_grafos}
		\caption[Aquitectura de Grafos]{
		(tomada de \cite{BoostGrafos)}
		%\label{F-dimensions-emotion}
	\end{figure}
	

	\section{C++}

	\section{MPI}

	\section{Boost}

	


	\section{Parallel Boost Graph Library}

	\begin{quotation}
		The Parallel Boost Graph Library is an extension to the Boost Graph Library (BGL) for parallel and distributed computing. It offers distributed graphs and graph algorithms to exploit coarse-grained parallelism along with parallel algorithms that exploit fine-grained parallelism, while retaining the same interfaces as the (sequential) BGL. Code written using the sequential BGL should be easy to parallelize with the parallel BGL. Visitors new to the Parallel BGL should read our architectural overview.\cite{wwwBoost} 
		\end{quotation} 

	\todo[inline,caption={TODO}]{Representacional}


\chapter{Virtualización}
\section{Virtualización de un cluster}


En este capitulo se presenta el procedimientos para la virtualización de un Ambiente de Pruebas que simula el funcionamiento de un centro de computación científica con el esquema de Apolo. 

Este proceso es importante para familizarse con el ambiente y los procedimientos en un ambiente seguro, con la posilidad de interacturar con los componentes de manera segura sin afectar de ninguna manera un centro de computo real. 

La siguiente es la descripción de los pasos a seguir para la creación de un ambiente de pruebas paralelo virtualizado que simula el clúster de Apolo.\footnote{Estas instrucciones fueron probadas en un ambiente de Linux Fedora 18.}

\subsection{Requerimientos Mínimos}

\begin{itemize}
	\item Verifique que su computador puede virtualizar: Ingrese a la BIOS y verifique que esta activada la opción de Virtualización de Hardware. 

	\item 3Gb de RAM (o mas)  disponibles para virtualizar.\footnote{1024 RAM para cada Maquina Virtual}
\end{itemize}

\subsection{Configuración del Nodo Maestro}
\begin{enumerate}

\item Descargue e Instale VirtualBox Manager junto con el correspondiente ``Extension Pack'' de la siguiente dirección: \url{https://www.virtualbox.org/wiki/Downloads} \footnote{Tanto la versión de VirtualBox como la versión del ``Extension Pack'' deben coincidir}

	\begin{figure}[H]
		\centering
		\includegraphics[width=0.5\textwidth]{aux/vb_instalado}
		\caption{Ventana de Virtualbox después de la instalación}
		%(tomada de \cite{WikiEmotion)}
		\label{vb_instalado}
	\end{figure}


\item Descargar la imagen del Master del siguiente enlace: \url{http://goo.gl/8eTJOr}

\item Una vez instalado VirtualBox en su computador, proceda a instalar el Extensión Pack: 
	
	\begin{figure}[H]
		\centering
		\includegraphics[width=0.5\textwidth]{aux/sinextensionpack}
		\caption{VirtualBox antes de instalar el ``Extension Pack''}
		%(tomada de \cite{WikiEmotion)}
		%\label{F-dimensions-emotion}
	\end{figure}
	
\begin{itemize}


	\item En VirtualBox acceda al menú: Archivo $\rightarrow$ Preferencias $\rightarrow$ Sección Extensiones $\rightarrow$ Proceda a instalar el ``Extension Pack''. 
	
	\begin{figure}[H]
		\centering
		\includegraphics[width=0.5\textwidth]{aux/conextensionpack}
		\caption{Despues de Instalar el ``Extension Pack''}
		%(tomada de \cite{WikiEmotion)}
		%\label{F-dimensions-emotion}
	\end{figure}
	

	\item Vaya a la sección Red $\rightarrow$ Adicione una red Sólo Anfitrión. Asegúrese en caso de tener o crear otra red ``Sólo anfitrión'' (Host Only) que todos los nodos compartan la misma red.


	
	\begin{figure}[H]
		\centering
		\includegraphics[width=0.5\textwidth]{aux/hostonly}
		\caption{Red Sólo Anfitrion ``vboxnet0''}
		%(tomada de \cite{WikiEmotion)}
		%\label{F-dimensions-emotion}
	\end{figure}
	
	

	\item Haga clic en aceptar para finalizar la operación.

\end{itemize}

\item En VirtualBox, importe desde el Menú $\rightarrow$ Importar Appliance. Deje la configuración por defecto y \textbf{no} reinicialice la dirección MAC.



\begin{figure}[H]
	\centering
	\includegraphics[width=0.5\textwidth]{aux/importappliance}
	\caption{Importar Maquina Virtual}
	%(tomada de \cite{WikiEmotion)}
	%\label{F-dimensions-emotion}
\end{figure}


\begin{figure}[H]
	\centering
	\includegraphics[width=0.5\textwidth]{aux/applianceops1}
	\caption{Opciones de configuración de la Maquina Virtual}
	%(tomada de \cite{WikiEmotion)}
	%\label{F-dimensions-emotion}
\end{figure}


\begin{figure}[H]
	\centering
	\includegraphics[width=0.5\textwidth]{aux/applianceops2}
	\caption{Más opciones de configuración de la Maquina Virtual}
	%(tomada de \cite{WikiEmotion)}
	%\label{F-dimensions-emotion}
\end{figure}



\item Seleccione la máquina virtual llamada ``Master'' y vaya a la configuración y revise la siguiente configuración en esta:

\begin{itemize}

	\item En la sección Sistema, pestaña Procesador, debe tener dos procesadores y habilitado PAE/NX.

	
	
	
	\begin{figure}[H]
		\centering
		\includegraphics[width=0.5\textwidth]{aux/procesadores}
		\caption{Procesadores e Inicio por PAE/NX}
		%(tomada de \cite{WikiEmotion)}
		%\label{F-dimensions-emotion}
	\end{figure}
	
	

	\item En la sección Red, pestaña Adaptador 1, deberá estar configurado como NAT. 


	
	\begin{figure}[H]
		\centering
		\includegraphics[width=0.5\textwidth]{aux/rednat}
		\caption{Configuración Red Adaptador 1: NAT}
		%(tomada de \cite{WikiEmotion)}
		%\label{F-dimensions-emotion}
	\end{figure}
	
	

	\item En la pestaña Adaptador 2 deberá estar en Adaptador Sólo--Anfitrión y el nombre deberá ser \texttt{vboxnet0}\footnote{Tenga en cuenta que este adaptador se llama \texttt{vboxnet0} en Linux, en Windows tendrá otro nombre, lo más importante es que sea la misma interfaz de red de Sólo Anfitrión, ya que sino dará lugar al problema de reasignación de interfaces de red dentro del nodo Master, en otras palabras, asignará las interfaces eth2 y eth3 en vez de asignar eth0 y eth1 a la NAT y a la Sólo Anfitrión respectivamente.}.


	
	\begin{figure}[H]
		\centering
		\includegraphics[width=0.5\textwidth]{aux/redhost}
		\caption{Red Host Only (Sólo Anfitrión)}
		%(tomada de \cite{WikiEmotion)}
		%\label{F-dimensions-emotion}
	\end{figure}
	
	

	\item Acepte todos los cambios.

\end{itemize}

\item Seleccione la máquina Master e iníciela.

\item Una vez iniciada la máquina virtual del Master proceda a crear una nueva máquina virtual como Nodo Trabajador a partir de los siguientes pasos:

\begin{itemize}
	\item Haga clic en Crear Nueva Máquina.

	\item El nombre será \texttt{compute-0-0}.

	\item El tipo será Linux.

	\item La versión será Red Hat de 64 bits.

	
	\begin{figure}[H]
		\centering
		\includegraphics[width=0.5\textwidth]{aux/opcnodo1}
		\caption{Opciones para el compute-0-0}
		%(tomada de \cite{WikiEmotion)}
		%\label{F-dimensions-emotion}
	\end{figure}
	
	\item Clic en siguiente.

	\item Asigne 1024 Megabytes de Memoria RAM\footnote{La cantidad de memoria RAM asignada será proporcional a la que tenga el sistema en el cual están las máquinas virtuales. Igual con el disco duro y los procesadores, sin embargo, para efectos de ver la paralelización en memoria compartida se recomienda tener 2 procesadores por máquina}.

	\item Cree el disco duro con 30 gigas de espacio, recuerde que esto se asignará dinámicamente. El tipo de disco duró será VDI y dinámicamente asignado.



	
	\begin{figure}[H]
		\centering
		\includegraphics[width=0.5\textwidth]{aux/nodohd}
		\caption{Crear un Nuevo Disco Virtual (VHD)}
		%(tomada de \cite{WikiEmotion)}
		%\label{F-dimensions-emotion}
	\end{figure}
	

	
	
	\begin{figure}[H]
		\centering
		\includegraphics[width=0.5\textwidth]{aux/nododvi}
		\caption{Tipo de disco a crear}
		%(tomada de \cite{WikiEmotion)}
		%\label{F-dimensions-emotion}
	\end{figure}
	


	
	\begin{figure}[H]
		\centering
		\includegraphics[width=0.5\textwidth]{aux/hddinamico}
		\caption{Opciones para Disco Dinámico o Estático}
		%(tomada de \cite{WikiEmotion)}
		%\label{F-dimensions-emotion}
	\end{figure}
	



	\begin{figure}[H]
		\centering
		\includegraphics[width=0.5\textwidth]{aux/nodohdsize}
		\caption{Tamaño del Disco}
		%(tomada de \cite{WikiEmotion)}
		%\label{F-dimensions-emotion}
	\end{figure}
		

\end{itemize}

\item Una vez creado el nodo trabajador se procede a configurarlo a partir de los siguientes pasos:

\begin{itemize}
	\item Señalar la máquina \texttt{compute-0-0} y dar clic en Configurar.

	\item En la pestaña Sistema, asegúrese de que tenga la cantidad de memoria RAM correcta

	\item Deshabilite el \texttt{floppy disk}

	\item habilite la red como dipositivo de \texttt{booteo} y además súbalo como primer dispositivo, sólo deberá quedar la tarjeta de red como primera opción y como segunda el disco duro de la máquina virtual, de esta manera nos aseguramos que esta máquina virtual pueda instalarse automáticamente de manera desatendida, por esta razón el clúster es escalable. 

	
	\begin{figure}[H]
		\centering
		\includegraphics[width=0.5\textwidth]{aux/nodoopsboot}
		\caption{Opciones del Nodo }
		%(tomada de \cite{WikiEmotion)}
		%\label{F-dimensions-emotion}
	\end{figure}
	


	\item En la pestaña Procesador asegúrese de que hay dos procesadores y está habilitado PAE/NX.



	\begin{figure}[H]
		\centering
		\includegraphics[width=0.5\textwidth]{aux/nodoprocesadores}
		\caption{Opciones Procesador para el Nodo}
		%(tomada de \cite{WikiEmotion)}
		%\label{F-dimensions-emotion}
	\end{figure}
		

	\item En la sección Red deberá configurar sólo la primera interfaz de red y deberá estar configurada como Sólo--Anfitrión y deberá tener \texttt{vboxnet0}.


	
	\begin{figure}[H]
		\centering
		\includegraphics[width=0.5\textwidth]{aux/nodored}
		\caption{Opciones de Red del Nodo}
		%(tomada de \cite{WikiEmotion)}
		%\label{F-dimensions-emotion}
	\end{figure}
	
	

	\item Acepte los cambios.
\end{itemize}


\item Repita los pasos para crear otro nodo si lo considera necesario para crear el \texttt{compute-0-1}, siempre y cuando la computadora que usted tiene soporte una tercera máquina virtual ejecutandose al tiempo que el Nodo Master y el \texttt{compute-0-0}

\item Ingrese en el Nodo Master como usuario \texttt{root} y la contraseña es \texttt{apolito123!}

\item Abra una consola (La combinación de teclas Alt+F2 permite ejecutar una aplicación escribiendo su nombre)



\begin{figure}[H]
	\centering
	\includegraphics[width=0.5\textwidth]{aux/terminal}
	\caption{Abrir Terminal}
	%(tomada de \cite{WikiEmotion)}
	%\label{F-dimensions-emotion}
\end{figure}



\item Para agregar las interfaces de los nodos ejecute el siguiente comando: 


 \begin{verbatim}
 $ insert-ethers
 \end{verbatim}


\begin{figure}[H]
	\centering
	\includegraphics[width=0.5\textwidth]{aux/insert}
	\caption{Captura de los nodos por parte del Master}
	%(tomada de \cite{WikiEmotion)}
	%\label{F-dimensions-emotion}
\end{figure}



\item Escoja \texttt{Compute}. Aparecerá una interfaz de consola mostrándole los nodos agregados en la medida que se les vaya asignando IP por medio de DHCP y una imagen de Linux para instalar con PXE.



\begin{figure}[H]
	\centering
	\includegraphics[width=0.5\textwidth]{aux/compute}
	\caption{inserción de los Nodos tipo ``Compute''}
	%(tomada de \cite{WikiEmotion)}
	%\label{F-dimensions-emotion}
\end{figure}



\item Encienda de uno en uno los Nodos trabajadores que haya creado, empezando por el \texttt{compute-0-0} y así sucesivamente. Observará que aparece en la interfaz \texttt{insert-ethers} la dirección MAC del Nodo Trabajador, se le asignará el nombre compute-0-0 y si aparece un símbolo de asterisco es porque recibió exitosamente la imagen para la instalación de Linux.




\begin{figure}[H]
	\centering
	\includegraphics[width=0.5\textwidth]{aux/nodosinsertados}
	\caption{Nodos ya reconocidos por el master}
	%(tomada de \cite{WikiEmotion)}
	%\label{F-dimensions-emotion}
\end{figure}



\item Una vez de que los nodos se termine de instalar automáticamente salga de la interfaz de \texttt{insert-ethers} en el Master con la tecla F8.

\item Ingrese al directorio especificado con el siguiente comando:


\begin{verbatim}
$ cd /export/apps/installers
\end{verbatim}


\item Descargue el instalador del comando \texttt{htop} con la siguiente instrucción:


\begin{verbatim}
$ wget http://goo.gl/TDWExw
\end{verbatim}


\item Instale \texttt{htop} en el nodo Master con la siguiente instrucción: 


\begin{verbatim}
$ rpm -ivh htop*.rpm}
\end{verbatim}


\item Ahora instale masivamente \texttt{htop} en el resto del clúster con el siguiente comando:


\begin{verbatim}
$  rocks run host ``rpm -ivh /share/apps/installers/htop*.rpm''}
\end{verbatim}


\item El cluster está completo.

\end{enumerate}

\subsection{Configuración de los Nodos Esclavos}

\begin{enumerate}
	\item Una vez que se tiene el nodo Master funcionando y por lo menos un nodo trabajador como el \texttt{compute-0-0} se procede a realizar los siguientes pasos de configuración:

	\begin{itemize}
	\item Adicione un usuario sin privilegios con los siguientes comandos\footnote{En adelante el usuario de ejemplo será smonsalve, pero usted podrá asignar el nombre de usuario que desee}:

	\begin{itemize}
		\item  Para crear un usuario sin privilegios: 


		\begin{verbatim}
		$ adduser smonsalve
		\end{verbatim}
		
		\item Para cambiar la contraseña del usuario:


		\begin{verbatim}
		$ passwd smonsalve
		\end{verbatim}
		

		\item Para sincronizar el usuario creado en todo el cluster, este usuarió estará creado tanto en el nodo master como en los nodos trabajadores.


		\begin{verbatim}
		$ rocks sync users
		\end{verbatim}
		
	\end{itemize}

	\item Para cambiar del usuario \texttt{root} al recient creado usuario sin privilegios: 

	\begin{verbatim}
	$ su - smonsalve
	\end{verbatim}

	\item Se le harán algunas preguntas de contraseñas, puede llenarlas o déjelas vacías y continuar presionando la tecla \texttt{enter} varias veces: 

	\end{itemize}
	
\end{enumerate}

\subsection{Apagar el Clúster}

Cada vez que se vaya a apagar el ambiente de Pruebas es necesario realizarlo con el siguiente procedimiento para evitar dañar la configuración de los nodos. 

\begin{itemize}
	\item Desde una consola en el master ejecutar el siguiente comando que le indicara los respectivos nodos que se apaguen de manera adecuada. 

	 \begin{verbatim}
	$ rocks run host poweroff
 	\end{verbatim}

 	\item Desde la misma consola para apagar el master: 

 	\begin{verbatim}
 	$ poweroff
 	\end{verbatim}
 	
  
\end{itemize}



\section{Posibles Problemas}

\begin{itemize}

	\item Los nodos de trabajo no inician Correctamente: Es posible que estos hayan sido apagados de manera incorrecta. 

	\begin{itemize}
		\item Remueva los nodos con el siguiente comando: 


		\begin{verbatim}
		$ insert-ethers --remove compute-0-0
		$ insert-ethers --remove compute-0-1
		\end{verbatim}

		\item Apagargregue de nuevo los nodos: 


		\begin{verbatim}
		$ insert-ehters
		\end{verbatim}
		
		
	\end{itemize}


\end{itemize}






\chapter{Instalación}
Ejemplo Programa Corriendo en Apolo

Programa Normal. 

Ejecturar


Instalacion de Boost


Repast HPC
wget http://repast.sourceforge.net/hpc_tutorial/SRC.tar.gz
tar xvf SRC.tar.gz

rocks run host rmpb -ivh /export/apps/installers/*.rpm


Agregar usuario sin privilegios. 

adduser smonsalve
passwd smonsalve
rocks sync users


insert-ethers --remove compute-0-0
insert-ethers --remove compute-0-1
insert-ethers 

rocks run host hostname
rocks run host poweroff



Debido a las relaciones de confianza entre el master y los nodo


ssh compute-0-0
ssh compute-0-1


Para monitorear la actividad del master y los nodos podemos utiliar el programa htop, el cual se encarga de mostrarnos los procesos, las cpus y los cores disponibles dentro del nodo. 



\chapter{Ejecución}

Ejemplo Programa Corriendo en Apolo

Programa Normal. 

Ejecturar


Ejecución

Contando ya con el ambiente necesario para la ejecucion del codigo a paralelizar se procede a realizar los siguientes pasos: 


desde la maquina virtual del master abrimos una consola. 

nos loggeamos como el usuario sin privilegios a través de ssh

smonsalve@192.168.56.101

% mpirun -n 4 -machinefile nodes.txt breadth_first_search < weighted_graph.gr

%weighted_graph-bfs.dot


nodes.txt


compute-0-0
compute-0-0
compute-0-1
compute-0-1




Codigo Fuente

Datos de Entrada

Salida ( a donde? )

Archivo de Errores

Torque  ( Nodos  )

PBS




Instalacion de Boost




rocks run host rmpb -ivh /export/apps/installers/*.rpm


Agregar usuario sin privilegios. 

adduser smonsalve
passwd smonsalve
rocks sync users


insert-ethers --remove compute-0-0
insert-ethers --remove compute-0-1
insert-ethers 

rocks run host hostname
rocks run host poweroff



Debido a las relaciones de confianza entre el master y los nodo


ssh compute-0-0
ssh compute-0-1


Para monitorear la actividad del master y los nodos podemos utiliar el programa htop, el cual se encarga de mostrarnos los procesos, las cpus y los cores disponibles dentro del nodo. 
 

\chapter{Conclusiones}
\begin{itemize}

	\item Con este proyecto se realizó una serie de procedimientos para simular un Ambiente de Pruebas que replica la arquitectura del centro de computación científica Apolo. Para trabajar en  este ambiente de pruebas fue necesario aplicar los conocimientos adquiridos en materias como Sistemas Operativos, Arquitectura de Computadores, Lenguajes de Programación, Telemática, Estructuras de datos y Algoritmos, entre otras.  

	\item La Computación Paralela es un campo que en nuestro contexto tiene un gran mercado para su aplicación y este trabajo permite una introducción al tema, el cual se espera sea aprovechado por otras personas para su aplicación. 

	\item Las librerías de Boost son  reconocidas por su alta calidad, las cuales brindan elementos para el Usuario final que facilitan y potencian el trabajo realizado, permitiendo un análisis complejo de gran cantidad de datos, mediante diferentes herramientas, agilizando la implementación de tales proyectos con una alta calidad. 

	\item Las librerías de Boost mediante su implementación basada en la programación genérica permite hacer uso  de sus librerías de una manera fácil y transparente para el programador, permitiendo de esta manera una codificación clara y elegante, lo cual repercute en el tiempo de desarrollo y su legibilidad. 

	\item MPI proporciona un encapsulamiento para las librerías de Boost que permite el funcionamiento del sistema, de tal manera que es transparente para el usuario la comunicación entre toda la arquitectura. 

\end{itemize}

\chapter{Trabajo Futuro}
Adicional a las posibilidades de uso de este proyecto planteadas en la Justificación y posibles aplicaciones, se desea continuar con este proyecto en aplicación a la solución de problemas reales. 

Dentro de las nuevas posibilidades se abren para continuar se destacan: 

\begin{itemize}
	\item Realizar un análisis de un grafo con gran cantidad de datos.
	
	\item Realizar un estudio del rendimiento de un programa con una implementación de Boost en Apolo.

	\item Agregar una sección para el manejo de colas en el clúster. (Torque o PBS )  
	
	\todo[inline,caption={TODO}]{referencias}

\end{itemize}

\chapter{Glosario}
	\label{chapGlosario}

	\begin{description}
		\item[HPC:] (High Performance Computing) Computación de alto desempeño.
		\item[Data Center:]
	\end{description}

\newpage

\bibliography{Tesis}

\bibliographystyle{plain}	
	aca
	\cite{czarnecki2000generative}
	\cite{wwwBoost}
	\cite{stroustrup2013c++}
	\cite{andrei2001modern}
	\cite{Wall2000}
	\cite{Boost}
	\cite{Karniadakis}
	\cite{Kernighan1988}

\newpage

\appendix

\chapter{Repositorio del código}

	Para una copia del código utilizado dirigirse a: \url{https://github.com/smonsalve/tesis.git}

	OVA de la maquina virtual con Rocks instalado y dos nodos.

	Input data para Grafo (X Vertices y Y Aristas).

	Makefile de Instalación.

\end{document}
