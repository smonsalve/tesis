\documentclass[ignorenonframetext]{beamer}
%\documentclass[utf8]{article}
%\usepackage[]{beamerarticle}
\usepackage[utf8x]{inputenc}
\usepackage[spanish]{babel}
\usepackage{times}
\usepackage[T1]{fontenc}
\usepackage{pgfpages}
\usepackage{graphics}
\usepackage{graphicx}


\mode<presentation>{
  \usebackgroundtemplate{\includegraphics[width=\paperwidth,height=\paperheight]{imgs/backgroundtemplate.jpg}} 
  \setbeamercovered{transparent}
  \usefonttheme{structurebold}
  \setbeamersize{sidebar width left=1cm}
  \setbeamersize{sidebar width right=0.5cm}
  \setbeamersize{description width right=0.5cm}
  \setbeamertemplate{blocks}[rounded][shadow=true]
%%%%%%%%%%%%%%%%%%%%%%%%%%%%%Experimental%%%%%%%%%%%%%%%%%%%%%%%%%%%%%%%%%%%%
 % \setbeamertemplate{headline}{\vspace{10cm}}
%%%%%%%%%%%%%%%%%%%%%%%%%%%%%%%%%%%%%%%%%%%%%%%%%%%%%%%%%%%%%%%%%%%%%%%%%%%%%
  \useinnertheme{rounded}
%\useinnertheme{rectangles}
  \useoutertheme{default}
%  \usecolortheme{rose}

  \setbeamertemplate{frametitle}
  {
    \begin{centering}
      \color{blue}
      \textbf{\insertframetitle}
      \par
    \end{centering}
  }

  \AtBeginPart{
    \begin{frame}
      \partpage
    \end{frame}
  }

  \AtBeginSection[]{
    \begin{frame}<beamer>
      \frametitle{Contenido}
      \tableofcontents[currentsubsection]
    \end{frame}
  }

  \setbeamertemplate{headline}
                    {%
                      \begin{beamercolorbox}[default theme]{section in head/foot}
                        \vskip40pt\vskip0pt
                      \end{beamercolorbox}%
                    }

}

%%%%%%%%%%%%%%%%%%%%%%%%%%%%%Experimental%%%%%%%%%%%%%%%%%%%%%%%%%%%%%%%%%%%%
% \newenvironment{changemargin}[2]{%
%   \begin{list}{}{%
%     \setlength{\topsep}{10pt}%
%     \setlength{\leftmargin}{#1}%
%     \setlength{\rightmargin}{#2}%
%     \setlength{\listparindent}{\parindent}%
%     \setlength{\itemindent}{\parindent}%
%     \setlength{\parsep}{\parskip}%
%   }%
%   \item[]}{\end{list}}
%%%%%%%%%%%%%%%%%%%%%%%%%%%%%%%%%%%%%%%%%%%%%%%%%%%%%%%%%%%%%%%%%%%%%%%%%%%%%

%%%%%%%%%%%%%%%%%%%%%%%%%%%%%Experimental%%%%%%%%%%%%%%%%%%%%%%%%%%%%%%%%%%%%
%\setlength{\hoffset}{1in}
%\setlength{\oddsidemargin}{0in} \setlength{\evensidemargin}{0in}
%\setlength{\voffset}{1in} \setlength{\topmargin}{0in}
%\setlength{\headheight}{1in}
%\setlength{\topmargin}{0.4in}

%El siguiente es el unico que funciona
%\setlength{\headsep}{0.4in}

%--
%\setlength{\textheight}{11in - TOPMARGIN - BOTTOMMARGIN - 0.7IN IF HEADER}
%\setlength{\textwidth}{8.5IN - LEFT MARGIN - RIGHT MARGIN}
%\setlength{\footskip}{-1in}
%\setlength{\footsep}{0.4in}
%%%%%%%%%%%%%%%%%%%%%%%%%%%%%%%%%%%%%%%%%%%%%%%%%%%%%%%%%%%%%%%%%%%%%%%%%%%%%

%%%%%%%%%%%%%%%%%%%%%%%%%%%%%Experimental%%%%%%%%%%%%%%%%%%%%%%%%%%%%%%%%%%%%

% \setbeamertemplate{headline}
% {%
%   \begin{beamercolorbox}[default theme]{section in head/foot}
%     \vskip18pt\insertnavigation{\paperwidth}\vskip0pt
%   \end{beamercolorbox}%
% }

% \setbeamertemplate{footline}
% {%
%   \begin{beamercolorbox}[infolines theme]{section in head/foot}
%     \vskip2pt\insertnavigation{\paperwidth}\vskip1pt
%   \end{beamercolorbox}%
% }
%%%%%%%%%%%%%%%%%%%%%%%%%%%%%%%%%%%%%%%%%%%%%%%%%%%%%%%%%%%%%%%%%%%%%%%%%%%%%

\title[Proyecto de Grado]{
  Instalación, Implementación y Uso de las librerías Boost y su aplicación en grafos para el centro de computación científica Apolo}
%\subtitle{}

\author[]{Sergio Andrés Monsalve--Castañeda \texttt{<smonsal3@eafit.edu.co>}
}

\institute[Universidad EAFIT]{
  %\large Centro de Computación Científica Apolo
}

\date[Universidad EAFIT]{2013} 

\subject{HPC, Apolo, Grafos}

\keywords{Recorrido, Grafos, Paralelo, HPC}


%% \AtBeginSubsection[]{
%%   \begin{frame}<beamer>
%%     \frametitle{Contenido}
%%     \tableofcontents[currentsection,currentsubsection]
%%   \end{frame}
%% }

%The following line must be commented if you won't need a slide per line
%\beamerdefaultoverlayspecification{<+->}

\begin{document}

\begin{frame}
 \begin{figure}[htp]
	\centering
	\includegraphics[width=0.38\textwidth]{../aux/logo_EAFIT}
\end{figure}
\end{frame}


\begin{frame}
  \titlepage
\end{frame}


\mode<all>
\section{Introducción}

\section{Ambiente de Pruebas}

\section{Instalación Software Necesario}

\section{Ejecución}


\begin{frame}
\frametitle{Conclusiones}


\end{frame}

\begin{frame}
\frametitle{Trabajo Futuro}


\end{frame}

\begin{frame}
\frametitle{Referencias}
	\begin{itemize}
	\item refs
	\item refs
	\item refs
	\end{itemize}
\end{frame}

\begin{frame}
\frametitle{Preguntas?}
      \begin{figure}[ht]
        \centering
        \includegraphics[scale=0.4]{imgs/gaz}
      \end{figure}
\end{frame}

\begin{frame}
\frametitle{Agradecimientos}

  \small 
  Quiero agradecer al profesor Juan Guillermo Lalinde quien desde el inicio de mi carrera me inspiró y apoyó en grandes retos académicos, por su paciencia y confianza a la hora de explicarme cada una de las múltiples dudas que surgían en el proceso.\\

  A Juan David Pineda por toda su paciencia para explicarme el funcionamiento de Apolo.\\

  Por ultimo manifestarle mi gratitud a John Jairo Silva, Alejandro Gómez, Mateo Gómez, Jaime Pérez y John Mario Gutiérrez por sus múltiples explicaciones y colaboraciones.

\end{frame}




\mode*

\input{partes/Sumario} 
% All of the following is optional and typically not needed.
\appendix
\section<presentation>*{\appendixname}
\subsection<presentation>*{For Further Reading}

\begin{frame}[allowframebreaks]
  \frametitle<presentation>{For Further Reading}

  \begin{thebibliography}{10}

  \beamertemplatebookbibitems
  % Start with overview books.

  \bibitem{Author1990}
    A.~Author.
    \newblock {\em Handbook of Everything}.
    \newblock Some Press, 1990.


  \beamertemplatearticlebibitems
  % Followed by interesting articles. Keep the list short.

  \bibitem{Someone2000}
    S.~Someone.
    \newblock On this and that.
    \newblock {\em Journal of This and That}, 2(1):50--100,
    2000.
  \end{thebibliography}
\end{frame}
 
\end{document}
